\subsection*{Pregunta 10 (Programación dinámica)}
\subsubsection*{Pseudocódigo}
Se agrega una resta luego de cada calculo de agrupación y división $
\Bar{w}=w-\mu$ y se utuliza el algpritmo de la pregunta 7.

\subsubsection*{}
Recibe: dos matrices de ceros y unos

Devuelve: una transformación entre las dos matrices y su peso
\begin{algorithmic}[1]
  \TITLE{\textsc{DPMinTransformMean}$(A, B)$}
  \STATE $m=A.size$
  \FOR{i=1 TO m}
    \STATE $matching[i],w_g=${\textsc{DPMinMatching}$(A[i], B[i])$}
    \STATE $w+=w_g$
  \ENDFOR
  \RETURN $matching,w$
   
\end{algorithmic}
\subsubsection*{}

Recibe: dos arreglos de ceros y unos

Devuelve: un matching de peso mínimo entre los dos arreglos, y el respectivo peso
\begin{algorithmic}[1]
  \TITLE{\textsc{DPMinMatchingMean}}$(A,B)$
  \STATE $blocksA = \textsc{GetBlocks$(A)$}$
  \STATE $blocksB = \textsc{GetBlocks$(B)$}$
  \STATE $m = blocksA.size$
  \STATE $n = blocksB.size$
  \STATE $matching[m][n]$
  \STATE $\mu = sum(blocksA[i])/sum(blocksB[j])$
  \FOR {i=1 TO m}
    \STATE $peso[i]=[n]$
    \FOR{j=1 TO n}
        \STATE $\textsc{DPMinMatchingMeanRec$(blocksA,blocksB,i,j,matching,peso,\mu)$}$
    \ENDFOR
  \ENDFOR
  \RETURN $peso[m][n]$
\end{algorithmic}
\subsubsection*{}
Recibe: dos arreglos de tamaños de bloques, la cantidad de bloques a usar, y la memoria

Devuelve: un matching entre los bloques usados de peso mínimo
\begin{algorithmic}[1]
  \TITLE{\textsc{DPMinMatchingMeanRec}}$(blocksA,blocksB,i,j,matching,peso, \mu)$
  \IF{$i=1$}
    \STATE $w=blocksA[1]/(sum(blocksB[1]..blocksB[j])$
    \STATE $matching[i][j].push((1,1)..(1,j))$
    \STATE $peso[i][j] = |w - \mu|$
  \ENDIF
  \IF{$j=1$}
    \STATE $w=sum(blocksA[1]..blocksA[i])/blocksB[1]$
    \STATE $matching[i][j].push((1,1)..(i,1))$
    \STATE $peso[i][j] = |w - \mu|$
  \ELSE
    \STATE $peso[i][j] = INF$
    \FOR{x=1 TO i-1}
        \STATE $w=peso[x][j-1]$
        \STATE $w=w+sum(blocksA[x+1]..blocksA[i])/blocksB[j]$
        \STATE $newMatching=matching[x][j-1]$
        \STATE $newMatching.push((x+1,j)..(i,j)$
        \STATE $\Bar{w} = |w - \mu|$
        \IF{$\Bar{w} < peso[i][j]$}
            \STATE $peso[i][j] = \Bar{w}$
            \STATE $matching[i][j] = newMatching$
        \ENDIF
    \ENDFOR
    \FOR{y=1 TO j-1}
        \STATE $w=peso[i-1][y]$
        \STATE $w=w+blocksA[i]/sum(blocksB[y+1]..blocksB[j])$
        \STATE $newMatching=matching[i-1][y]$
        \STATE $newMatching.push((i,y+1)..(i,j)$
        \STATE $\Bar{w} = |w - \mu|$
        \IF{$\Bar{w} < peso[i][j]$}
            \STATE $peso[i][j] = \Bar{w}$
            \STATE $matching[i][j] = newMatching$
        \ENDIF
    \ENDFOR
    \ENDIF
\end{algorithmic}
\subsubsection*{Análisis}
El tiempo de ejecución es igual que el de la pregunta 7 $O(pq^2)$, ya que solo se agrega una resta al peso cada vez que se calcula el peso de una división o una agrupación,
\subsubsection*{Implementación}
La implementación del algoritmo en C++ se puede encontrar en el siguiente \href{https://github.com/Guillermo598/ProyectoADA/blob/master/Pregunta10.cpp}{link}.