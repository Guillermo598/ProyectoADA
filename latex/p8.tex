\subsection*{Pregunta 8 (Lectura de imágenes)}
\subsubsection*{Pseudocódigo}
Lee una imagen y como arrays de chars, cada componente R, G y B es representado por un char. Luego estos chars son convertidos a ints y se multiplica con una constante para calcular la luminansia y se compara con el umbral y se define si es 1 o 0.

\subsubsection*{}
Recibe: una imagen, el umbral, y las tres constantes de R, G y B

Devuelve: una matriz de 0 y 1
\begin{algorithmic}[1]
  \TITLE{\textsc{getMatrixFromPNG}$(file, umbral, R, G, B)$}
  \STATE $image = load(file)$
  \STATE $matrix[height][width]$
  \FOR{i=1 TO height}
    \FOR{j=1 TO width}
        \STATE $pixelOffset = image + (j + width * i) * 3$
        \STATE $Y = R*pixelOffset[0] + G*pixelOffset[1] + B*pixelOffset[2]$
        \IF{Y $>$ umbral}
            \STATE $vec[j] = 0$
        \ELSE 
            \STATE $vec[j] = 1$
        \ENDIF
    \ENDFOR
    \STATE $matrix[i] = vec$
  \ENDFOR
  \RETURN $matrix$
   
\end{algorithmic}
\subsubsection*{Implementación}
La implementación del algoritmo en C++ se puede encontrar en el siguiente \href{https://github.com/Guillermo598/ProyectoADA/blob/master/Pregunta8.cpp}{link}.