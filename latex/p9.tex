\subsection*{Pregunta 9 (Animación)}

\subsubsection*{Pseudocódigo}
Se calcula el promedio de cada pixel de amnas imagenes para crear la imagen resultante, y se recibe un parametro extra para definir cual algoritmo se quiere utilizar para realizar el matching.
\subsubsection*{}
Recibe: Dos imágenes y un parámetro para elección de algoritmo

Devuelve: El peso del matching y la transformación de la imagen
\begin{algorithmic}[1]
  \TITLE{\textsc{transformImages}$(imageA, imageB, method)$}
  \STATE $matrixA = load(imageA)$
  \STATE $matrixB = load(imageB)$
  \IF{$method == 0$}
    \STATE $peso = GreedyMinTransform(matrixA, matrixB, transform)$
  \ELSIF{$method == 1$}
    \STATE $peso = DPMinTransform(matrixA, matrixB, transform)$
  \ELSIF{$method == 2$}
    \STATE $peso = DPMinTransformMean(matrixA, matrixB, transform)$
  \ENDIF
  \STATE $\textsc{getTransition(imageA, imageB)}$
  \RETURN $peso$
\end{algorithmic}

\subsubsection*{}
Recibe: Dos imágenes

Devuelve: Una transformación de imagen
\begin{algorithmic}[1]
  \TITLE{\textsc{getTransition}$(imageA, imageB)$}
  \STATE $A = load(imageA)$
  \STATE $B = load(imageB)$
  \FOR{$i=0$ TO $A.size$}
    \STATE $transition[i] = (A[i] + B[i])/2$
  \ENDFOR
  \STATE $write\_jpg(transition)$
\end{algorithmic}

\subsubsection*{Implementación}
La implementación del algoritmo en C++ se puede encontrar en el siguiente \href{https://github.com/Guillermo598/ProyectoADA/blob/master/Pregunta9.cpp}{link}.