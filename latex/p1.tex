\subsection*{Pregunta 1 (Voraz)}
\subsubsection*{Pseudocodigo}
Primero, presentamos un subalgoritmo que calcula los bloques de los arreglos en tiempo lineal, ya que recorre dicho arreglo una sola vez. Cabe notar que este subalgoritmo se usará en todos los algoritmos futuros. Abajo de este se encuentra el algoritmo voraz principal.
\subsubsection*{}
Recibe: un arreglo de ceros y unos

Devuelve: un arreglo de tamaños de los bloques
\begin{algorithmic}[1]
  \TITLE{\textsc{GetBlocks}$(array)$}
  \STATE $i = 1$
  \WHILE{$i < array.size$}
    \WHILE{$array[i] == 0$}
        \STATE $i = i + 1$
    \ENDWHILE
    \STATE $size = 0$
    \IF{$array[i] == 1$}
        \WHILE{$array[i] == 1$}
            \STATE $size = size + 1$
            \STATE $i = i + 1$
        \ENDWHILE
        \STATE $blocks.push(size)$
    \ENDIF
  \ENDWHILE
  \RETURN $blocks$
\end{algorithmic}
\subsubsection*{}
Recibe: dos arreglos de ceros y unos

Devuelve: un matching entre los dos arreglos y su peso
\begin{algorithmic}[1]
  \TITLE{\textsc{GreedyMinMatching}$(A, B)$}
  \STATE $blocksA = \textsc{GetBlocks$(A)$}$
  \STATE $blocksB = \textsc{GetBlocks$(B)$}$
  \STATE $m = blocksA.size$
  \STATE $n = blocksB.size$
  \IF{$m > n$} 
    \STATE $max_i = ratio = m/n$
    \STATE $j = 1$
    \FOR{$i=1$ TO m}
        \IF{$i \leq max_i$}
            \STATE $num = num + blocksA[i]$
        \ELSE
            \STATE $w = w + num/blocksB[j]$
            \STATE $num = blocksA[i]$
            \STATE $max_i = max_i + ratio$
            \STATE $j = j + 1$
        \ENDIF
        \STATE $matching.push(i,j)$
    \ENDFOR
    \STATE $w = w + num/blocksB[j]$
  \ELSE
    \STATE $max_j = ratio = n/m$
    \STATE $i = 1$
    \FOR{$j=1$ TO n}
        \IF{$j \leq max_j$}
            \STATE $den = den + blocksB[j]$
        \ELSE
            \STATE $w = w + blocksA[i]/den$
            \STATE $den = blocksB[j]$
            \STATE $max_j = max_j + ratio$
            \STATE $i = i + 1$
        \ENDIF
        \STATE $matching.push(i,j)$
    \ENDFOR
    \STATE $w = w + blocksA[i]/den$
  \ENDIF
  \RETURN $matching, w$
\end{algorithmic}

\subsubsection*{Analisis}
Dejando de lado el cálculo de los bloques de cada arreglo, el tiempo de ejecución del algoritmo propuesto es de $O(max\{m,n\})$. Esto se debe al condicional de la linea 5. Si $m$, la cantidad de bloques del arreglo $A$, es mayor, se itera asignándole un solo \textit{matching} a cada bloque de $A$. En el caso contrario, se hace lo mismo pero con los bloques de $B$. El calculo del peso del \textit{matching} $w$ se hace durante este proceso, por lo que solo le aumenta tiempo constante a la iteración. 


\subsubsection*{Implementación}
La implementación del algoritmo en C++ se puede encontrar en el siguiente \href{https://github.com/Guillermo598/ProyectoADA/blob/master/Pregunta1.cpp}{link}.